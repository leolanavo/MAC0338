\documentclass[a4paper, 12pt]{article}

\usepackage[brazilian]{babel}
\usepackage[utf8]{inputenc}
\usepackage[T1]{fontenc}
\usepackage[a4paper]{geometry}
\usepackage{amsmath}
\usepackage{amssymb}
\usepackage{indentfirst}
\usepackage{graphicx}
\usepackage{hyperref}
\usepackage{float}
\usepackage{csquotes}
\usepackage{listings}
\renewcommand{\rmdefault}{ptm}
\usepackage{xcolor}
\usepackage{inconsolata}

\lstset{
    language=bash, %% Troque para PHP, C, Java, etc... bash é o padrão
    basicstyle=\ttfamily\small,
    numberstyle=\footnotesize,
    numbers=left,
    backgroundcolor=\color{gray!10},
    frame=single,
    tabsize=2,
    rulecolor=\color{black!30},
    title=\lstname,
    escapeinside={\%*}{*)},
    breaklines=true,
    breakatwhitespace=true,
    framextopmargin=2pt,
    framexbottommargin=2pt,
    inputencoding=utf8,
    extendedchars=true,
    literate={á}{{\'a}}1 {ã}{{\~a}}1 {é}{{\'e}}1,
}


\title{Lista 3 - MAC0338}
\author{Leonardo Lana Violin Oliveira - 9793735}
\date{27 de Agosto de 2017}

\begin{document}
\maketitle
%%%%%%%%%%%%%%%%%%%%%%%%%%%%%%%%%%%%%%%%%%%%%%%%%%%%%%%%%%%%%%%%%%%%%%%%%%%%%%%%%%%%%
\section*{2}
%%%%%%%%%%%%%%%%%%%%%%%%%%%%%%%%%%%%%%%%%%%%%%%%%%%%%%%%%%%%%%%%%%%%%%%%%%%%%%%%%%%%%
\subsection*{A}
O número esperado de comparações na linha 6 é $E[X]$, onde $X$ é o número de vezes que
o elemento $i$ do vetor não é o maior elemento do subvetor $v[1...i]$, então podemos
afirmar que $X = \sum_{i = 1}^{n} X_i$, onde
$$ X_i = 
\begin{cases}
1 , \text{se o elemento \textit{i} do subvetor \textit{v[1...i]} não é o maior
elemento deste subvetor} \\
0, cc
\end{cases}
$$

Já que a probabilidade é uniforme, então temos que a probabilidade de um elemento não
ser o maior elemento do subvetor de tamanho $i$ é $Pr\{X_i = 1\} = 1 - \dfrac{1}{i}$,
logo:

\begin{align*}
E[X] &= \sum_{i = 0}^{n} E[X_i] = \sum_{i=0}^{n} Pr\{X_i = 1\} = \\
	 &= \sum_{i = 0}^{n} 1 - \dfrac{1}{i} = 
     \sum_{i = 0}^{n} 1 - \sum_{i = 0}^{n} \dfrac{1}{i} = \\
     &= n - \sum_{i = 0}^{n} \dfrac{1}{i} > n - (\ln n) = \mathcal{O}(n) \\
     &\text{\textbf{OU}}\\
     &= n - \sum_{i = 0}^{n} \dfrac{1}{i} < n - (\ln n + 1) = \Omega(n)
\end{align*}

Logo, $E[X] = \Theta(n)$
%%%%%%%%%%%%%%%%%%%%%%%%%%%%%%%%%%%%%%%%%%%%%%%%%%%%%%%%%%%%%%%%%%%%%%%%%%%%%%%%%%%%%
\subsection*{B}

O número esperado de atribuições feitas na linha 7 é $E[Y]$, onde Y é o número de vezes
que o elemento $i$ do vetor é o menor elemento do subvetor $v[1...i]$, então podemos
afirmar que $Y = \sum_{i = 0}^{n} Y_i$, onde

$$ Y_i = 
\begin{cases}
1, \text{se o elemento $i$ do vetor é o menor elemento do subvetor $v[1...i]$}\\
0, cc
\end{cases}
$$

Já que a probabilidade é uniforme, então temos que a probabilidade de um elemento
ser o menor elemento do subvetor de tamanho $i$ é $Pr\{Y_i = 1\} = \dfrac{1}{i}$, logo:

\begin{align*}
E[Y] &= \sum_{i = 0}^{n} E[Y_i] = \sum_{i = 0}^{n} Pr\{Y_i = 1\} = \\
	 &= \sum_{i = 0}^{n} \dfrac{1}{i} = \Theta(\lg n)
\end{align*}

Então o valor esperado para o número de atribuições na linha 7 é $\Theta(\lg n)$
%%%%%%%%%%%%%%%%%%%%%%%%%%%%%%%%%%%%%%%%%%%%%%%%%%%%%%%%%%%%%%%%%%%%%%%%%%%%%%%%%%%%%
\section*{8}
\begin{lstlisting}[mathescape=true]
Mediana(V, c, f)
    t <- f - c + 1
    pos <- (f + c)/2
    Se t%2 != 0, então
        Retorna V[$\lceil$pos$\rceil$]
    senão,
        Retorna (V[pos] + V[pos + 1])/2

SAmediana (V1, c1, f1, V2, c2, f2)
    m1 <- Mediana(V1, c1, f1)
    m2 <- Mediana(V2, c2, f2)
    t <- f1 - c1 + 1
    pos <- (f1 + c1)/2

    Se t = 2, então
        Vm = Intercala(V1, c1, f1, V2, c2, f2, 4)
        Retorna Mediana(Vm, 1, 4)

    senão se m1 > m2, então
        Se t%2 = 0, então
            Retorna (SAmediana(V1, c1, pos+1, V2, pos, f2))
        senão,
            Retorna (SAmediana(V1, c1, $\lceil$pos$\rceil$, V2, $\lceil$pos$\rceil$, f2))
    
    senão se (m1 < m2), então
        Se t%2 = 0, então
            Retorna (SAmediana(V1, pos, f1, V2, c2, pos+1))
        senão,
            Retorna (SAmediana(V1, $\lceil$pos$\rceil$, f1, V2, c2, $\lceil$pos$\rceil$))
    
    senão,
    Retorna m1

\end{lstlisting}

%%%%%%%%%%%%%%%%%%%%%%%%%%%%%%%%%%%%%%%%%%%%%%%%%%%%%%%%%%%%%%%%%%%%%%%%%%%%%%%%%%%%%
\section*{9}
Para entendimento da descrição, vamos chamar a função \textit{caixa-preta} de
\textit{mediana}. E usaremos a função \textit{PARTICIONE} do \textit{QUICKSORT}.

Dado um vetor, usaremos a função \textit{mediana} para obter o valor da mediana,
e procuraremos a posição deste valor no vetor ($\mathcal{O}(n)$), e chamamos o 
\textit{PARTICIONE} usando o índice da mediana, para particionar o vetor em duas 
metades, a primeira metade menor que a mediana e a segunda metade é maior que a
mediana.

Comparamos o $k$ com a posição da mediana, se os valores forem iguais, o
algoritmo encerra, se $k$ menor que a posição, a função será chamada
recursivamente para a primeira metade do vetor, caso $k$ seja o valor maior, a
recursão será chamada para o segunda metade do vetor.
%%%%%%%%%%%%%%%%%%%%%%%%%%%%%%%%%%%%%%%%%%%%%%%%%%%%%%%%%%%%%%%%%%%%%%%%%%%%%%%%%%%%%

\end{document}
