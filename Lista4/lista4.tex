\documentclass[a4paper, 12pt]{article}

\usepackage[brazilian]{babel}
\usepackage[utf8]{inputenc}
\usepackage[T1]{fontenc}
\usepackage[a4paper]{geometry}
\usepackage{amsmath}
\usepackage{amssymb}
\usepackage{indentfirst}
\usepackage{graphicx}
\usepackage{hyperref}
\usepackage{float}
\usepackage{csquotes}
\usepackage{listings}
\renewcommand{\rmdefault}{ptm}
\usepackage{xcolor}
\usepackage{inconsolata}

\lstset{
    language=bash, %% Troque para PHP, C, Java, etc... bash é o padrão
    basicstyle=\ttfamily\small,
    numberstyle=\footnotesize,
    numbers=left,
    backgroundcolor=\color{gray!10},
    frame=single,
    tabsize=2,
    rulecolor=\color{black!30},
    title=\lstname,
    escapeinside={\%*}{*)},
    breaklines=true,
    breakatwhitespace=true,
    framextopmargin=2pt,
    framexbottommargin=2pt,
    inputencoding=utf8,
    extendedchars=true,
    literate={á}{{\'a}}1 {ã}{{\~a}}1 {é}{{\'e}}1,
}


\title{Lista 4 - MAC0338}
\author{Leonardo Lana Violin Oliveira - 9793735}
\date{27 de Agosto de 2017}

\begin{document}
\maketitle
%%%%%%%%%%%%%%%%%%%%%%%%%%%%%%%%%%%%%%%%%%%%%%%%%%%%%%%%%%%%%%%%%%%%%%%%%%%%%%%%%%%%%
\section*{8}

Nesta resolução usaremos uma função similar ao \textit{CountingSort}, com duas
mudanças, não há mais a ordenação no loop final da função e esta nova função,
chamaremos ela de \textit{Counting}, retorna o vetor com as soma acumulativa
(número de elementos que há entre a primeira posição do vetor e a
\textit{i}-ésima posição do mesmo).

Para a consulta basta pega este vetor com a soma acumulativa e subtrair o
conteúdo da \textit{a}-ésima com o da \textit{b}-ésima, ou seja, sendo
\textit{v} o vetor da soma acumulativa, a nossa consulta deveria retornar
$v[b] - v[a]$.
%%%%%%%%%%%%%%%%%%%%%%%%%%%%%%%%%%%%%%%%%%%%%%%%%%%%%%%%%%%%%%%%%%%%%%%%%%%%%%%%%%%%%
\end{document}
